\section{K. Hauser, O. Ramon - Generalization of the Capture Point to Nonlinear Center of Mass and Uneven Terrain \cite{RamonHauser2015}}
Year: 2015
\subsection*{Summary}
This paper generalizes the concept of Capture Point as we know it based on a LIP model on a flat terrain. The new model is a Nonlinear Inverted Pendulum, meaning that the height of the CoM is no more constrained to be constant but it can be a steep lin or a parabola. In the mean time also uneven terrains are considered, when characterized by a polygonal shape.
\subsection*{Key points / Takeaways}
See the Journal Club of 2016 on this paper.
\subsection*{Weak points}
\subsection*{Ideas}e or a parabola. In the mean time also uneven terrains are considered, when characterized by a polygonal shape.
\subsection*{Key points / Takeaways}
See the Journal Club of 2016 on this paper.
\subsection*{Weak points}
\subsection*{Ideas}
\begin{itemize}
\item extend to non-zero momentum case;
\item extend to more complex environments;
\item extend to more complex CoM trajectories;
\end{itemize}
