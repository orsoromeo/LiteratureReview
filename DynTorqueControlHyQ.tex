\section{Focchi, Semini et al. - Dynamic Torque Control of a Hydraulic Quadruped Robot \cite{SeminiFocchi}}
Authors: Thiago Boaventura, Claudio Semini, Jonas Buchli, Marco Frigerio, Michele Focchi, Darwin G. Caldwell\\
year: 2012
\subsection*{Summary}
Main features required to a quadruped robot controller:
\begin{itemize}
\item precise: able to perform accurate footholds placement
\item fast: able to produce fast movements
\item robust: needed against uncertainties and disturbancies
\item compliant: needed to withstand impacts and crashes
\end{itemize}
The \textbf{whole-body control} is a model-based controller which operates through an inversion of the dynamics, this can be seen as a \textbf{force/torque control}. The \textbf{position control} will still be present as a lower level control but will have low gains and thus have a little influence on the lower higher level.\\
When we talk about stability in the field of dynamic motions we had better refer to \textbf{limit-cicle stability}.
\subsection*{Key points / Takeaways}
\begin{itemize}
\item the controller can be described as a three-layer controller:
\begin{itemize}
\item The outer layer is a position control in the joint space
\item the middle layer is a force/torque controller
\item the inner layer is a position/speed control in the operational space and a pressure controller
\end{itemize}
All these quantities are linearized thanks to a feedback linearization which leads to a linearized input $v$, this is computed as a PID controller. This PID controller is also responsible for the compliance of the system: in order to keep safe the mechanical parts and the electronics from impacts the torque peaks are always kept at safe level underneath 160 Nm.
\item The robustness of the controller is not only achieved by means of robust controllers but also with robust hardware. The flow of the valves must have a large bandwith so that it can meet the requirements of different gaits.
\item An equation for force control for hydraulically actuated robots can be computed using the \textit{Bernoulli} equation (force equilibrium) \cite{ThiagoSemini}. 
\end{itemize}
\subsection*{Weak Points}
\subsection*{Ideas}