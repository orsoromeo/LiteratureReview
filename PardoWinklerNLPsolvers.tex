\section{D. Pardo, A. Winkler - Evaluating direct transcription and nonlinear optmization methods for robot motion planning}

Analized integration methods:
\begin{itemize}
\item Newton step
\item Direct collocation
\begin{itemize}
\item Trapezoidal method
\item Cubic Hermite polynomial
\end{itemize}
\item Implicit methods
\begin{itemize}
\item Runge Kutta 4
\end{itemize}
\end{itemize}

Analized solvers:
\begin{itemize}
\item Sequential Quadratic Programming (SQP) sover: SNOPT
\item Interior Point Method (IPM): IPOPT
\end{itemize}

Influent parameters:
\begin{enumerate}
\item number of discretization nodes
\item number of control inputs and states
\item sparsity
\item variables initialization
\item number of constraints
\end{enumerate}

Evaluation quantities:
\begin{enumerate}
\item computation time (CPU time)
\item quality of the solution (cost function value)
\item accuracy of the solution: absolute value of the feedback action in the simulation and in the hardware
\end{enumerate}
In \textbf{Trajectory Optimization (TO)} in general we always need to verify that the optimal solution is coherent with the dynamic constraints.
\subsection*{Conclusions}
\begin{itemize}
\item The choice of the solver is very determinant for the speed of the computation of the optimal solution
\item The integration scheme (Cubic Hermit rather than trapezoidal) is very determinant on the computation speed for a high number of integration nodes (above 20)
\item The initialization of the variables is fundamental for the computation time and for finding a feasible solution. Naive approaches like random or zero initialization might not work. A good strategy is homotopy (i.e. incremental solutions).
\end{itemize}