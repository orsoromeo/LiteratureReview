\section{Raibert - Running on four legs as though they were one -1986}

\subsection*{Summary}

In this paper the concept of \textbf{Virtual leg} is introduced. A virtual leg is a fictitious leg which mirrors the result of the locomotion performed by two or more legs which are in the same time on the ground.

\subsection*{Take aways}

\begin{itemize}
\item The control scheme implemented in this work, as much as Raibert's preceeding work on the one-legged hopper, is based on the idea to decouple the three main quantities of interest for locomotion: the robot's \textbf{pitch} angle, the \textbf{forward speed} and the \textbf{height} of each single jump.

\item The forward velocity and step lenght are adjusted making use of the so called \textbf{neutral point}, which is the point on the ground that, if the foothold is placed in that very same point, produces a constant speed (neither acceleration nor deceleration)

\item \textbf{One-foot gaits} are defined as those gaits which can be performed with the use of one single virtual leg. These gaits can be performed theoretically with any number of legs, but they must all be coordinated in such a way to obtain the behavior of one single leg.
\end{itemize}