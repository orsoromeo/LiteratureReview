\section{D.E. Koditschek - Legged Self-manipulation \cite{johnson}}
Authors: D.E. Koditschek and Aaron M. Johnson\\
Year: 2013
\subsection*{Summary}
\subsection*{Key points / Takeaways}
Self-manipulation is a distinguished branch of locomotion which aims at exploiting the well mature field of manipulation. For this reason self-manipulation differs from both traditional manipulation and from locomotion for a few reasons; the main difference from traditional manipulation are the following:
\begin{itemize}
\item in self-manipulation the robot is the object and in the same time it is also the palm; therefore these two reference frames (palm and object) are in this case coincident. This means that it's the robot manipulating itself in order to perform a given movement.
\item we're mainly concerned with the motion of the robot rather than the object.
\item in self-locomotion the dynamics of legs and trunk are not decoupled.
\end{itemize}
Self-manipulation neither coincides with locomotion though, in that locomotion usually aims at using \textit{templates}the simplified dynamic models. Self-manipulation instead aims at finding the best \textit{anchors} which exploit the actuators performances.\\
In this sense self-manipulation can be used to \textit{anchor} and locomotion problem solved in the traditional lumped-mass based method.
\subsection*{Weak points}
\subsection*{Ideas}
