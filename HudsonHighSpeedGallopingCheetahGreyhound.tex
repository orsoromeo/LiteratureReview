\section{Hudson - High speed galloping in the cheetah (Acinonyx jubatus) and the racing greyhound (Canis familiaris): spatio-temporal and kinetic characteristics - 2012}


\subsection*{Summary}

\begin{itemize}
\item both cheetah and greyhounds use rotary gallop at fast speeds ($>14 m s^{-1}$). It is still unclear why the rotary gallop is preferred to transverse gallop, probably because of the reduced momentum losses during stance. The rotary gallop may become transverse gallop during abrupt direction changes (front legs are used to give yaw direction). 

\item There is a time-constant (short) aerial phase between the \textit{leading hind} leg and the \textit{trailing front} leg which is not dependent from speed. There is then another aerial phase which increases with speed and takes place between the \textit{trailing front} leg and the \textit{leading hind} leg. During this aerial phase the cheetah/greyhound extends his trunk. This phase gets longer with increased speed and it thus appears only at higher speeds. At slow speeds therefore the stance of the \textit{leading hind leg} and the stance of the \textit{trailing front leg} can be almost coincident, in this case this gate is named \textbf{canter} (or \textbf{three-beat transverse gallop}).

\item Cheetahs use significantly shorter duty factors than greyhounds (i.e. cheetahs have shorter stance times).  Cheetahs swing time slightly decreases with speed while greyhounds swing time slightly increases. In general we can say that the swing time keeps constant with speed (between $200$ and $300 ms$)

\item cheetahs increase the peak of the \textbf{hind legs} contact forces with speed (to compensate the decrease of the stance time and keep the total impulse constant) while the \textbf{front legs} contact forces do not increase more than $900N$, probably because of muscles limits.

\item the maximum speed is limited by a mix of:
\begin{enumerate}
\item swing time limit
\item ground speed matching
\item muscular power available for external work
\item peak force that the limbs can withstand
\item maintaining stability
\item mental motivation of the animal 
\end{enumerate}

\item the body weight is not symmetrically distributed on the four legs. At steady state 56 \% of the weight is on the front legs, 27 \% is charged on the \textbf{trailing hind leg} and 17\% on the \textbf{leading hind leg}. During accelerations instead more than 50\% of the weight is supported by the hind legs.
\end{itemize}