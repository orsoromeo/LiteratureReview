\section{K. Hauser - Motion Planning for legged and humanoid robots \cite{Hauser2008}}
year: 2008
\subsection*{Summary}
The author defines the existence of different \textit{modes} $\sigma$ which belong to the same family $\Sigma$. The dimension of $\Sigma$, $dim(\Sigma)$, is lower than the dimension of the configuration space $Q$. All the modes of the same family are non-intersecting each other, meaning that if the robot is moving on a given trajectory $q$ which is on the mode $\sigma_1$ and wants to move to the mode $\sigma_2$ (of the same family $\Sigma$), it will have to go through another family $F$ which interesects the family $\Sigma$.
\subsection*{Key points / Takeaways}
\subsection*{Weak points}
\begin{itemize}
\item The author distinguishes between \textit{volume reducing constraints} and \textit{dimensionality reducing contraints}. And how can we classify the nonholonomic contraints then?
\end{itemize}

\subsection*{ideas}